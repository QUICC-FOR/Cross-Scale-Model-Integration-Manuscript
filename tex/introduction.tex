\section*{Introduction}

Models of species range limits have wide applications and can play a large role in conservation biology, where they can be used as decision-making tools in biodiversity management \citep{Guisan2013}.
Due to large temporal and spatial scales as well as the complex and nonlinear nature of ecosystem dynamics, it is often impossible to construct experiments that can adequately explore the ecological processes generating species range limits \citep{Wu1995, Levin1998}. 
Hence, range models are essential tools that have been applied to a large number of ecological subfields, including biogeography \citep{Schurr2012}, invasion biology \citep{Catterall2012, Gallien2012}, evolution \citep{Jay2012}, hybrid zone dynamics \citep{Engler2013}, and the impacts of climate change on species distributions \citep{Thuiller2011, Blois2013, Thuiller2014}. 
However, despite having recognized potential to support decision making, the contributions of these models to conservation and applied ecology remain limited \citep{Guisan2013}.

Two important constraints on modeling ecological processes prevent models from making unbiased predictions with acceptable uncertainty: (1) having the appropriate ecological theory needed to link data to modeling objectives, and (2) having sufficient data over a range of conditions to maintain coherence between the spatial and temporal scales of data and theory.
Another constraint is a paucity of data over wide ranges of ecological conditions and scales.
Poor data or incoherent models will result in predictions with unacceptably high uncertainty or precise but biased predictions.
In recent decades, however, modeling techniques have proliferated to take advantage of what is available in terms of both data and modeling platforms.
Model diversification also arises due to the diverse processes generating species ranges \citep{Soberon2007}.
In the context of forecasting, which is the focus here, mechanistic and correlative approaches represent quite different modes of inference, but both have strong limitations. 
Fine-scale mechanistic models may capture important ecological processes quite well, but due to their specificity, they may perform poorly when applied at the scale of species ranges.
For instance, biotic interactions (including trophic interactions) are usually not modeled mechanistically at the scale of species distributions because they are not well-known, or have not been recorded, despite being considered a key determinant of range limits \citep{Holt2005, Soberon2007, Roux2012, Guo2013, Pigot2013}. 
On the other hand, coarse-scale correlative models statistically relating species occurrences to associated distributions of other variables at the same scale have the advantage of indirectly accounting for underlying processes \citep{Guisan2000}.
However, their predictions rely on the constancy of the relationships between occurrences and explanatory variables in time and space, implying that the selected variables are related to the processes limiting species ranges and that their correlations are constant for calibration and projection ranges in space and time \citep{Dormann2007}. 
Extrapolating beyond the scope of the original data (e.g., predicting ranges based on future climate) is therefore problematic, because nonlinearities in responses to novel combinations of the explanatory variables cannot be accommodated in models that do not simulate the underlying processes.
In cases where a selected explanatory variable indirectly accounts for a process (e.g., where the strength of biotic interactions are correlated with an environmental variable), extrapolation will produce biased predictions in temporal and spatial domains where the process and the environmental variable are decoupled.
Furthermore, correlative models often do not explicitly incorporate ecological processes into their formulations, and when applied at coarse scales, they may miss important fine-scale processes, making a multi-scale approach desirable when accurate forecasting is required \citep{Austin2011, Thuiller2013}.
On the other hand process-based models often require a variety of assumptions which may be poorly understood across space and time.

\subsection*{Toward an integrated approach for modeling ranges}
To integrate knowledge about species ranging from the stand to the landscape, we present an application of hierarchical Bayesian methods for integrating the predictions of multiple models.
Our approach is an alternative to other approaches to drawing inference from multiple models that offers improved flexibility in incorporating data and theory from multiple scales and provides transparent estimation of uncertainty.
The challenge is to account for all this available knowledge while producing a single prediction, potentially mitigating the limitations inherent in each modeling approach and contributing to more robust predictions \citep{Pearson2003, Guisan2005, Araujo2006, Quillet2010}.
Some researchers have explicitly reproduced the hierarchical nature of ecological systems using hierarchical models \citep[e.g.][]{Royale2008, Catterall2012, Strigul2012, Stewart-Koster2013, Soranno2014}. 
Recent literature has also focused on the development of hybrid models, which allow for flexible combinations between mechanistic and phenomenological sub-models \citep{Gallien2010, Franklin2010, Thuiller2013}. 
For instance, correlative models are used to account for abiotic variables limiting species distributions \citep{Guisan2005}, while mechanistic approaches can include space and time dynamics and therefore account for dispersal processes \citep{Kearney2008}. 
This complementarity has been used to merge the two types of models into hybrid integrated models \citep[e.g.][]{Keith2008, Anderson2009, Smolik2010, Naujokaitis2013, Boulangeat2014}. 
However, the link between different sub-models is based on assumptions about the scaling of ecological processes that are not easy to test \citep{Gallien2010}, and uncertainties are very difficult to quantify and identify from its many different sources. 
Moreover, hybrid approaches are not suitable for merging models that are based on the same processes but with different underlying hypotheses.

A second approach is to directly combine predictions.
If models operate at the same scales and have compatible parameterizations, predictions can be combined using multi-model inference \citep[e.g., model averaging, ensemble forecasting;][]{Thuiller2004, Araujo2007}. 
If models are not comparable because they are based on different data or hypotheses, the use of convergent predictions from multiple models is a way to provide more robust projections \citep{Morin2009, Marmion2009, Serra-Diaz2013}.
However, the applicability of ensemble forecasts are limited; for example, it is not currently possible to evaluate the effects of convergent predictions on total uncertainty of outcomes.
One of the greatest challenges inherent in both approaches above is estimating uncertainty, which is of utmost importance in a prediction context.
A more comprehensive understanding of uncertainty can guide biodiversity management and prioritize future data collection by identifying parameters that make large contributions to the total uncertainty in the model predictions \citep{McMahon2011}. 
In models that encompass multiple scales or levels of organization, it is particularly challenging to evaluate all sources of uncertainty, especially those related to model specification (\citealt{Calder2003} but see \citealt{Conlisk2013}).
Clearly, an approach that integrates predictions and uncertainty from multiple model types is needed \citep{Beck2012, Thuiller2013}. 
Such an approach would incorporate multiple modes of inference and integrate data from multiple sources and scales \citep{Levin1992, Peters2004, Thuiller2013}.

Here, we present a framework for modeling species range dynamics that aims to account for all the information available on a focal species, from a variety of sources and scales.
Contrary to hybrid and hierarchical models, the aim is not to link different sub-models into a single model, but to condition the predictions of a meta-model at the target scale (e.g., an entire species' range) with information from sub-models at a variety of spatial scales, allowing as much flexibility as possible regarding the type of information included. 
We use the power and generality of a hierarchical Bayesian framework, which allows us to include multiple data sources and modes of inference \citep{Clark2005, VanOijen2005, Clark2006, Hobbs2011, Hartig2012}. 
Another advantage of Bayesian methods is that uncertainty in model outputs is intuitive to interpret and reflects uncertainty at all levels of organization \citep{Clark2005, Cressie2009, Hobbs2011}. 
We illustrate our approach with two examples.
First, we use simulated data to present a complete example of the application of the framework to multiple information sources that are relevant at very different scales.
In a second example, we apply the framework to combine predictions of a correlative model relating the range of sugar maple (\emph{Acer saccharum}) to climate with the predictions of a phenological model, Phenofit, with the goal of improving uncertainty estimation when projecting model predictions to future climate.
