
\section*{Integrated modelling framework}
\aa{Integrated modelling framework and examples: I would just merge these two sections into: 'Examples of the integrated modelling framework'}

The key idea of our approach is to constrain the parameters of a correlative meta-model using predictions made by one or several mechanistic sub-models from other spatial scales.
\aa{Does the meta-model need to be correlative? Because later on you say that it 'could take the form of a correlative model' as implying that it could take other forms.}
The role of the metamodel is to integrate data at the same ecological scale as predictions. 
As an example, the metamodel could take the form of a simple correlative \ac{SDM}.
Ideally, each sub-model incorporates different data types and hypotheses that specify underlying ecological mechanisms contributing to the model output.
\tf{Check in examples if hypothesis are given}
\mt{they are there, though perhaps not so explicit}
The output of sub-models should provide additional predictions at a comparable ecological scale at which the metamodel operates, although it is not necessary for the sub-models to actually operate at the same scale.
For sub-models operating at a substantially different scale than the metamodel, scaling functions will be required; however, as we will show in the first example, these functions can be simple while still contributing useful information to the metamodel. 
All predictions may therefore be integrated in a single Bayesian framework when evaluating the parameters of the metamodel (see Box 1, Fig. \ref{fig:diagram}).
We present the complete framework in a hypothetical example where experimental results are  introduced as prior knowledge of the fundamental niche in a species distribution model (SDM).
A more formal presentation of the modelling framework appears in Box 1.
We follow with a second example where we apply our approach to integrate phenological information with occurrence data for a widespread North American tree.


\begin{figure}


%\documentclass{article}
%\usepackage{tikz}
%\usepackage{tikz-cd}
%\usepackage{amsmath}

%\usetikzlibrary{calc, shapes}
%\usetikzlibrary{shapes.geometric,shapes.arrows,decorations.pathmorphing}
%\usetikzlibrary{matrix,chains,scopes,positioning,arrows,fit}
%\begin{document}

%% --------------------------------------------------------------------------------------
%% ------part 1--------------------------------------------------------------------------
%% --------------------------------------------------------------------------------------

\begin{tikzpicture}
	
\matrix(m) [matrix of nodes, column sep=-0.5em,
	row sep=2em,
	minimum width=4em,
	minimum height=2em,
	column 1/.style={anchor=west, align=left, text width=10em},
	multi/.style={rectangle split,rectangle split parts=2}]
	{
	% zeroth line
	% blank space 
	&
	\textbf{Correlative Metamodel}
	&
	&
	\textbf{Mechanistic sub-model}
	\\
	% first line
	Data % m 1-1
	&
	$X_{M}, D_{M}$ % m 1-2
	&
	&
	$X_{S}, D_{S}$ % m 1-4
	\\
%second line
	|[multi]| Process % m 2-1 
	\nodepart{second}
	(prior and likelihood)
	&
	|[multi]|$p(\theta_{M})$ % m 2-2
	\nodepart{second}
	$p(X_{M} \mid \theta_{M}, D_{M})$
	&
	&
	|[multi]|$p(\theta_{S})$
	\nodepart{second}
	$p(X_{S} \mid \theta_{S}, D_{S})$
	\\

%third line
	
	Posterior
	&
	$p(\theta_{M} \mid X_{M}, D_{M})$
	&
	&
	$p(\theta_{S} \mid X_{S}, D_{S})$
	\\

%fourth line
	Prediction
	&
	$\psi_N = f(\theta_M, D_M)$
	&
	&
	$\psi_S = g(\theta_S, X_S, D_S)$
	\\

%fifth line
	Data
	&
	&
	&
	$\psi_S$
	\\
%sixth line
	|[multi]| Process
	\nodepart{second}
	(integrated likelihood)
	&
	$p(\theta_M \mid X_M, D_M)$
	&
	&
	$p(\psi_S \mid \theta_M)$
	\\


%seventh line
	Integrated posterior
	&
	&
	$p(\theta_M \mid \psi_S, X_M, D_M)$
	&
	\\

%eighth line
	Integrated prediction
	&
	&
	$\psi_I = f(\theta_M, D_M)$
	&
	\\
}; %end matrix

% The names of the nodes are automatically generated in the previous matrix. Since the
% matrix was named ``m'', all nodes have the name m-row-column

% metamodel
\draw [->] (m-2-2.south) -- (m-3-2.north);
\draw [->] (m-3-2.south) -- (m-4-2.north);
\draw [->] (m-4-2.south) -- (m-5-2.north);

%%mechanistic model
\draw [->] (m-2-4.south) -- (m-3-4.north);
\draw [->] (m-3-4.south) -- (m-4-4.north);
\draw [->] (m-4-4.south) -- (m-5-4.north);

%%separation line
\draw [line cap=rect, transform canvas={yshift=-1em}] (m-5-1.south west) -- (m-5-4.south east);
%(\linewidth-\pgflinewidth,0); 

\tikzstyle{opt}=[gray,dashed,rounded corners];
\draw [->,style=opt] (m-4-2.west) to[bend right=50] (m-7-2.west);
\draw [->,style=opt] (m-5-4.south) to (m-6-4.north);

%%integrated model
\draw [->] (m-7-2.south) -- (m-8-3.north west);
\draw [->] (m-6-4.south) -- (m-7-4.north);
\draw [->] (m-7-4.south) -- (m-8-3.north east);
\draw [->] (m-8-3.south) -- (m-9-3.north);


\end{tikzpicture}


\caption{The parameters of a correlative metamodel model (left column) are conditioned on the predictions of a mechanistic sub-model (right column).
The metamodel ($\theta_M$) operates at a single ecological scale and uses occurrence data (\(X_M\)) and explanatory variables ($D_M$) to produce a naive (i.e., not conditioned on sub-models) prediction $\psi_N$.
The mechanistic sub-model \(\theta_S\) includes data about the response (\(X_S\)) of lower-level behaviours of the system to explanatory variables ($D_S$). 
The models are integrated by calibrating $\theta_M$ to data ($X_M, D_M$) as well as the output of the sub-model ($\psi_S$). 
This is possible because predictions from the sub-model ($\psi_S$) emerge at the scale of the metamodel via a scaling function \(g(\theta_S, D_S)\).
The final prediction can be obtained by applying the integrated model to the original explanatory variables (as shown) or, if projection is desired, on some new set of explanatory variables (e.g., future climate).
This prediction incorporates multiple sources of information coming from several calibration datasets (i.e., $X_M, D_M, X_S, $ and $D_S$) as well as from multiple types of models (i.e., $\theta_M$ and $\theta_S$).
}
\label{fig:diagram}
\end{figure}
