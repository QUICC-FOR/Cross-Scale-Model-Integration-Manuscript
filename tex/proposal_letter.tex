\documentclass[11pt]{letter}
\usepackage[margin=1.5in]{geometry}

\makeatletter
\renewcommand{\closing}[1]{\par\nobreak\vspace{\parskip}%
  \stopbreaks
  \noindent
  \ifx\@empty\fromaddress\else
  \hspace*{\longindentation}\fi
  \parbox{\indentedwidth}{\raggedright
       \ignorespaces #1\\[1\medskipamount]%
       \ifx\@empty\fromsig
           \fromname
       \else \fromsig \fi\strut}%
   \par}
\makeatother

\usepackage{fontspec} 
\defaultfontfeatures{Mapping=tex-text} % converts LaTeX specials (``quotes'' --- dashes etc.) to unicode
\newopentypefeature{Contextuals}{NoAlternate}{-calt} % enable turning off the crazy Q in Garamond Premier Pro
\usepackage{xunicode}
\setmainfont{Garamond Premier Pro}
\setsansfont[Scale=0.9]{Optima Regular} 

\usepackage{wallpaper}

\signature{Matthew V. Talluto, \\
	Isabelle Boulangeat, \& \\
	Dominique Gravel \\
	Université du Québec à Rimouski \\ 
	mtalluto@gmail.com}
\address{Matthew V. Talluto, Ph.D. \\
	Départament du Biologie, Chimie, et Géographie \\
	300, Allée des Ursulines \\
	Rimouski, Québec G5L 3A1, Canada \\
	mtalluto@gmail.com}

\begin{document}
\addfontfeatures{Contextuals=NoAlternate}   % turn off the crazy Q in Garamond Premier Pro
\ThisULCornerWallPaper{0.22}{uqar-logo2}

\begin{letter}{Marcel Holyoak \\ 
	Editor-in-chief, \emph{Ecology Letters} \\ 
	Department of Environmental Science and Policy \\
	University of California, Davis \\
	One Shields Avenue \\
	Davis, CA 95616, USA }

\opening{Dear Dr. Holyoak,}

%TC:break opening
We are pleased to submit a proposal for a manuscript titled ``Cross-scale integration of knowledge for predicting species ranges: a metamodeling framework''. 
We would appreciate your consideration of our manuscript for the \emph{Ideas and Perspectives} section of \emph{Ecology Letters}.


%TC:break body
We present a novel framework for integrating small-scale, more mechanistic models into large-scale correlative species distribution models.
The processes generating species ranges are complex and operate at multiple scales; thus, a wealth of theory (e.g., metapopulation theory, niche theory, and complexity theory) underlies models of species ranges.
Furthermore, there is great interest in modeling ranges, resulting in a diversity of approaches.
In practice, however, most models do not include all available information; for example, correlative models predicting presence/absence as a function of climate often ignore smaller-scale processes such as growth, fecundity, and dispersal.
Furthermore, individual models often produce distinct predictions for the same species, with no simple way to reconcile these predictions.
Our contribution is the first to propose a conceptual and analytical framework that goes further than \emph{a posteriori} aggregation of predictions from different models. 
It draws on the flexibility of hierarchical Bayesian methods to integrate knowledge of a focal species at varying ecological scales in order to provide more robust predictions of its future distribution.
Moreover, our method quantifies uncertainty and its variablity in space while also propagating uncertainty from each information source.
This is of crucial for identifying knowledge gaps, as thus as a to guide future research.
Finally, under our framework future information can be added as it becomes available, which makes its potential application for adaptive management very appealing.
We illustrate our approach using two examples, and demonstrate that the framework can reduce both bias and uncertainty when projecting beyond the range of the original data.
We believe that our manuscript represents an important step toward a more holistic approach for forecasting species ranges.
This is of broad interest due to the importance of range shifts for theoretical and applied ecology, and we hope you will consider it for the \emph{Ideas and Perspectives} section.

% Suggestion from Isa for a novelty statement:


%TC:break closing
Thank you for considering our submission. We will be ready to submit the manuscript within one month. We look forward to your reply.

\closing{Sincerely,}

\end{letter}
\end{document}