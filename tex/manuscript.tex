\documentclass[11pt]{article}

%% ----------------------------------
%
%     Packages to be used in the final manuscript
%
%% ----------------------------------


\usepackage[margin=1in]{geometry}

% use proper unicode fonts
\usepackage[T1]{fontenc}
\usepackage[utf8]{inputenc}

\usepackage{amsmath} % for better display of equations

\usepackage{natbib} 

\usepackage{titling} % controls the way the title information is displayed
\pretitle{\begin{flushleft}\Large}
\posttitle{\end{flushleft}}
\predate{}
\postdate{}
\preauthor{\begin{flushleft}}
\postauthor{\end{flushleft}}
\setlength{\droptitle}{-3em}

\usepackage{authblk} % adds some nice options for displaying the author list

\usepackage{acronym} % set up some acronyms
\acrodef{SDM}{species distribution model}
\acrodef{MCMC}{Markov Chain Monte Carlo}

%% graphics packages
\usepackage{graphicx}
\usepackage{pdflscape}   % for figures in landscape orientation

% Tikz libraries for building the diagram
\usepackage{tikz}
\usepackage{tikz-cd}
\usetikzlibrary{calc, shapes}
\usetikzlibrary{shapes.geometric,shapes.arrows,decorations.pathmorphing}
\usetikzlibrary{matrix,chains,scopes,positioning,arrows,fit}



%% ----------------------------------
%
%     Packages to be used for drafts and editing
%     Remove from the final manuscript
%
%% ----------------------------------
%\usepackage[top=1in, left=1in, bottom=1in, right=2.5in]{geometry}
%
%\usepackage[textsize=tiny, backgroundcolor=white, textwidth=2.2in, colorinlistoftodos=true]{todonotes} % for margin notes using \todo{}
%
%% available colors:
%% red, green, blue, cyan, magenta, yellow, gray, drakfray, lightgray, brown, lime, olive, orange, pink, purple, teal, violet
%
%\newcommand{\mt}[1]{\todo[color=blue!40]{\textsuperscript{MT}#1}}
%\newcommand{\ib}[1]{\todo[color=magenta!40]{\textsuperscript{IB}#1}}
%\renewcommand{\aa}[1]{\todo[color=violet!40]{\textsuperscript{Aitor A}#1}}
%\newcommand{\ia}[1]{\todo[color=blue!20]{\textsuperscript{Isabelle A}#1}}
%\newcommand{\db}[1]{\todo[color=violet!20]{\textsuperscript{Dominique B}#1}}
%\newcommand{\rd}[1]{\todo[color=orange!40]{\textsuperscript{Ronnie D}#1}}
%\newcommand{\mjf}[1]{\todo[color=green!60]{\textsuperscript{Marie-Josée F.}#1}}
%\newcommand{\tf}[1]{\todo[color=yellow!60]{\textsuperscript{Tony F.}#1}}
%\newcommand{\dm}[1]{\todo[color=lime!50]{\textsuperscript{Dan M.}#1}}
%\newcommand{\ks}[1]{\todo[color=green!40]{\textsuperscript{Kevin S}#1}}
%\newcommand{\ns}[1]{\todo[backgroundcolor=red!100, linecolor=orange!100, bordercolor=cyan!100]{{\color{magenta!40}\textsuperscript{Strigul}#1}}}
%\newcommand{\wt}[1]{\todo[color=violet!40]{\textsuperscript{Wilfried T}#1}}
%\newcommand{\dg}[1]{\todo[color=pink!40]{\textsuperscript{DG}#1}}

% uncomment to disable individual authors
% \renewcommand{\mt}[1]{\relax}%
% \renewcommand{\ib}[1]{\relax}%
% \renewcommand{\aa}[1]{\relax}%
% \renewcommand{\ia}[1]{\relax}%
% \renewcommand{\db}[1]{\relax}%
% \renewcommand{\rd}[1]{\relax}%
% \renewcommand{\mjf}[1]{\relax}%
% \renewcommand{\tf}[1]{\relax}%
% \renewcommand{\dm}[1]{\relax}%
% \renewcommand{\ks}[1]{\relax}%
% \renewcommand{\ns}[1]{\relax}%
% \renewcommand{\wt}[1]{\relax}%
% \renewcommand{\dg}[1]{\relax}%

%% ----------------------------------
%
%     Title and authorship information
%
%% ----------------------------------


\title{Cross-scale integration of data and knowledge for predicting species ranges}
\date{}
\author[1,2]{Matthew V. Talluto}
\author[1,2]{Isabelle Boulangeat}
\author[3]{Aitor Ameztegui}
\author[4]{Isabelle Aubin}
\author[1,2,5]{Dominique Berteaux}
\author[1,2]{Alyssa Butler}
\author[10]{Dominique Cyr} %??
\author[10]{Fred Doyon} %??
\author[6]{C. Ronnie Drever}
\author[7]{Marie-Josée Fortin}
\author[1]{Tony Franceschini}
\author[8]{Jean Liénard}
\author[4]{Dan McKenney}
\author[2,3]{Kevin A. Solarik}
\author[8]{Nick Strigul}
\author[9]{Wilfried Thuiller}
\author[1,2]{Dominique Gravel}
\affil[1]{Département de biologie, Université du Québec à Rimouski, Rimouski, Quebec, Canada}
\affil[2]{Quebec Centre for Biodiversity Science, Montreal, Quebec, Canada}
\affil[3]{Centre d'Étude de la Forêt, Département des sciences biologiques, Université du Québec à Montréal, Montreal, Quebec, Canada}
\affil[4]{Great Lakes Forestry Centre, Canadian Forest Service, Natural Resources Canada, Sault Ste Marie, Ontario, Canada}
\affil[5]{Centre for Northern Studies, Université du Québec à Rimouski, Rimouski, Quebec, Canada}
\affil[6]{The Nature Conservancy Canada, Ottawa, Ontario, Canada}
\affil[7]{Department of Ecology and Evolutionary Biology, University of Toronto, Toronto, Ontario, Canada}
\affil[8]{Department of Mathematics, Washington State University Vancouver, Washington, USA}
\affil[9]{Laboratorie d'Ecologie Alpine, Unité Mixte de Recherche, Université de Grenoble, Grenoble, France}
\affil[10]{Affiliations to be filled in}


%% ----------------------------------
%
%     END PREAMBLE
%
%% ----------------------------------

\begin{document}

%% ----------------------------------
%
%     TITLE PAGE
%
%% ----------------------------------

%\listoftodos
\maketitle

\begin{flushleft}
\textbf{Keywords:} Uncertainty, scaling, range dynamics, trees, disturbances, patterns and processes, predict or understand, climate change, biotic interactions, spatial ecology, decision making, metamodelling, species distribution modeling
\end{flushleft}

\begin{abstract}
There is great interest in modeling species ranges, and such models are useful in guiding conservation and management decisions.
There presently exists a great diversity of modeling approaches, and individual models often use only a small subset of the total available information about a species.
For example, a broad-scale correlative model might use climate variables to predict presence or absence, but ignore what is known about smaller-scale processes such as the effect of local climate on growth, fecundity, and dispersal rates.
A central problem with the diversity of approaches is that multiple models produce differing predictions for the same organism, with no simple way to reconcile these predictions.
We present a flexible framework for integrating models at multiple scales using hierarchical Bayesian methods. 
The resulting meta-model produces probabilistic estimates of species presence with uncertainty that reflects error from all sub-models and data sources. 
These predictions reflect all of the information used as input for the original sub-models. 
We illustrate the approach through two examples, and demonstrate that the framework can substantially reduce uncertainty when projecting beyond the range of some of the original data sources.
Finally, we discuss the application of our method and its accessibility to conservation biologists and land managers.
Although the method requires extensive statistical programming experience, we anticipate the results will be of wide application and interest, and we therefore encourage collaboration between modelers and practitioners in applying our framework.
\end{abstract}

%% ----------------------------------
%
%     REST OF DOCUMENT
%
%% ----------------------------------


\section*{Introduction}
\aa{I like it very much now, it's good and concise. The only thing is that the first and last sentences in the first paragraph seem contradictory. Do models of species range limits play a large role on conservation biology or do their contributions remain limited? Maybe in the first sentence you wanted to say they ‘can’ play a large role? }
\aa{In addition, you may want to add this reference: Snell, R.S., Huth, A., Nabel, J.E.M.S., Bocedi, G., Travis, J.M.J., Gravel, D., Bugmann, H., Gutiérrez, A.G., Hickler, T., Higgins, S.I., Scherstjanoi, M., Reineking, B., Zurbriggen, N. \& Lischke, H. (2014) Using dynamic vegetation models to simulate plant range shifts. Ecography, In press.}

Models of species range limits have wide applications and play a large role in conservation biology, where they can be used as decision-making tools in biodiversity management \citep{Rosenzweig2008, Thuiller2008, Rodenhouse2009}.
\wt{I would just put Guisan et al. 2013 here. }
\dm{Just a note to you that I still think people confuse or associate potential distribution (range?) modeling with actual distribution models. One could write a paper on the understanding, use and mis-use of these terms.  I also think people often associate a purpose to a piece of work that mis-interprets the intended purpose }
Due to large temporal and spatial scales as well as the complex and nonlinear nature of ecosystem dynamics, it is often impossible to construct experiments that can adequately explore the ecological processes generating species range limits \citep{Wu1995, Levin1998}. 
Hence, range models are essential tools that have been applied to a large number of ecological subfields, including biogeography \citep{Schurr2012}, invasion biology \citep{Catterall2012, Gallien2012}, evolution, hybrid zone dynamics \citep{Engler2013}, and the impacts of climate change on species distributions \citep{Rosenzweig2008, Thuiller2008, Milad2011, Blois2013, Loyola2013}. 
\wt{after evolution above: Jay, F., Manel, S., Alvarez, N., Durand, E., Thuiller, W., Holderregger, R., Taberlet, P. \& François, O. (2012) Forecasting changes in population genetic structure of Alpine plants in response to global warming. Molecular Ecology, 21, 2354-2368.}
\wt{The above mixes up subfields vs applications}
\wt{for climate change citations:
Thuiller, W., Lavergne, S., Roquet, C., Boulangeat, I. \& Araujo, M.B. (2011) Consequences of climate change on the Tree of Life in Europe. Nature, 470, 531-534. \\
Thuiller, W., Pironon, S., Psomas, A., Barbet-Massin, M., Jiguet, F., S., L., Pearman, P.B., Renaud, J., Zupan, L. \& Zimmermann, N.E. (2014) The European functional tree of bird life in face of global change. Nature Communications, 5, 3118, DOI: 10.1038/ncomms4118.
}
However, despite having recognized potential to support decision making, the contributions of these models to conservation and applied ecology remain limited \citep{Dawson2011, Guisan2013}.
\wt{remove Dawson}

An important constraint on modelling ecological processes is having the appropriate ecological theory needed to link data to modelling objectives. 
Another constraint is the lacking of data over a while range of conditions.
Poor data or poorly conceived models will result in predictions with unacceptably high uncertainty, or, more dangerously, precise but biased predictions.
\tf{Why is it dangerous? 
It depends also of the importance of the bias. Some regression technique exists in order to deal multicolinearities and their basis is to produce precise and biased estimates. In this case, it is a way to reduce uncertainty. See e.g. ridge regression. With this kind of regression, there is no “danger”, and the estimates are known to be biased (and therefore should be interpreted and used as biased estimates).
Plus, I feel that the word dangerous is not the correct word. There is ho danger in this, no harm. 
}
In recent decades, however, modelling techniques have proliferated to take advantage of what is available in terms of both data and modeling platforms.
\dm{Again just a note that I think people often do not realize the nature/problems with data. Weather station data is a classic example…lots of issues remain hence spatial clikmate models have problems; species observation data may have many issues from mis-identifcation to inaccurate geocoding. Maybe some of this this should be noted more explicitly later in the manuscript but I am not married to this … I don’t want to dwell on it}
Model diversification also arises due to the diverse processes generating species ranges \citep{Soberon2007}.
In the context of forecasting, which is the focus here, mechanistic and correlative approaches represent quite different modes of inference, but both have strong limitations. 
Fine-scale mechanistic models may capture important ecological processes quite well, but due to their specificity, they may perform poorly when applied at the scale of species ranges.
For instance, biotic interactions (including trophic interactions) are usually not modelled mechanistically at the scale of species distributions because they are not well-known, or have not been recorded, despite being considered a key determinant of range limits \citep{Soberon2007, Roux2012, Guo2013, Pigot2013}. 
\mjf{add citation: Holt et al 2005}
\mjf{Holt, R. D., T.H. Keitt, M.A. Lewis, B.A. Maurer and M.L. Taper. 2005. Theoretical models of species’ borders: single species approaches. Oikos 108: 18-27.}
On the other hand, coarse-scale correlative models statistically relating species occurrences to associated distributions of other variables at the same scale have the advantage of indirectly accounting for underlying processes \citep{Guisan2000}.
However, their predictions rely on the constancy of the relationships between occurrences and explanatory variables in time and space, implying that the selected variables are related to the processes limiting species ranges and that their correlations are constant for calibration and projection ranges in space and time \citep{Dormann2007}. 
Extrapolating beyond the scope of the original data (e.g., predicting ranges based on future climate) is therefore problematic, because nonlinearities in responses to novel combinations of the explanatory variables cannot be accommodated in models that do not simulate the underlying processes.
Furthermore, correlative models often do not explicitly incorporate ecological processes into their formulations, and when applied at coarse scales, they may miss important fine-scale processes, making a multi-scale approach desirable when accurate forecasting is required \citep{Austin2007, Austin2011, Soranno2014}.
\wt{remove all 3 cites, add \\
Thuiller et al. 2013 ELE}
\dm{add to end: On the other hand process-based models often require assumptions, many of which are poorly understood across space and time. }

\subsection*{Toward an integrated approach for modelling ranges}
A key problem with the diversity of approaches is that multiple models often produce differing predictions for the same organism, with no simple way to reconcile these predictions.
\mjf{delete first sentence of this paragraph, it was already said above}
\aa{Change to: \\
In addition, an assortment of information (in the form of both data and theory)...}
\aa{from here}
\mjf{begin with: To integrate knowledge about species ranging from the stand to the landscape, we present an application...}
We present an application of hierarchical Bayesian methods for integrating the predictions of multiple models.
Our approach is an alternative to other approaches to drawing inference from multiple models that offers improved flexibility in incorporating data and theory from multiple scales and provides transparent estimation of uncertainty.
\ia{this is repetitive with the last paragraph of this section. \\
delete everything up and and including "For instance..."}
For instance, correlative models are often criticized because they assume that the selected explanatory variables limit (directly or indirectly) the species range, and, as mentioned above, that the relationship between the selected variables and the species occurrences will not change in the prediction context \citep{Araujo2006, Berteaux2006, Braunisch2013}.
\aa{to here}
However, additional information (in the form of both data and theory) is often available on the species of interest (e.g., growth rates in different climatic conditions, phenology, interactions with other species, sensitivity to disturbances) \citep{Holt2009, Thuiller2013}. 
The challenge is to account for all this available knowledge while producing a single prediction.
\dm{change to: while producing predictions---I do have as bit of difficulty with this notion. Multiple predictions would seem to me to be quite reasonable given the range of understanding and issues with data.  Reasonable people may differ if their understanding and beliefs hence multiple predictions that represent these may be desired. I am not sure my suggested wording in this sentence or the next is satisfactory.}
Integrating multiple models to provide a single prediction would allow more knowledge to be used, potentially mitigating the limitations inherent in each modelling approach and contributing to more robust predictions \citep{Pearson2003, Guisan2005, Araujo2006, Quillet2010}.

One approach is to integrate multiple models in a single modelling framework \citep{Buckley2010}. 
Some researchers have explicitly reproduced the hierarchical nature of ecological systems using hierarchical models \citep[e.g.][]{Royale2008, Catterall2012, Strigul2012, Stewart-Koster2013, Soranno2014}. 
Recent literature has also focused on the development of hybrid models, which allow for flexible combinations between mechanistic and phenomenological sub-models \citep{Gallien2010, Franklin2010, Thuiller2013}. 
For instance, correlative models are used to account for abiotic variables limiting species distributions \citep{Guisan2005}, while mechanistic approaches can include space and time dynamics and therefore account for dispersal processes \citep{Kearney2008}. 
This complementarity has been used to merge the two types of models into hybrid integrated models \citep[e.g.][]{Keith2008, Anderson2009, Smolik2010, Boulangeat2014}. 
\mjf{add citation: Naujokaitis-Lewis et al., 2013}
\mjf{Naujokaitis-Lewis, I.R., J.M.R. Curtis, L. Tischendorf, D. Badzinski, K. Lyndsay, M.-J. Fortin. 2013. Uncertainties in coupled species distribution–metapopulation dynamics models for risk assessments under climate change. Diversity and Distributions, 19: 541-554.}
However, the link between different sub-models is based on additional assumptions that are not easy to test \citep{Gallien2010}. 
\aa{which assumptions? not clear}
\ia{a bit vague here. Give an example?}
Moreover, hybrid approaches are not suitable for merging models that are based on the same processes but with different underlying hypotheses.
\ia{an example?}

A second approach is to directly combine predictions.
If models operate at the same scales and have compatible parameterizations, predictions can be combined using multi-model inference \citep[e.g., model averaging, ensemble forecasting;][]{Araujo2007, Diniz-Filho2009}. 
\wt{add: Thuiller, W. (2004) Patterns and uncertainties of species' range shifts under climate change. Global Change Biology, 10, 2020-2027.\\
remove: Diniz-Filho2009}
If models are not comparable because they are based on different data or hypotheses, the use of convergent predictions from multiple models is a way to provide more robust projections \citep{Morin2009, Marmion2009, Serra-Diaz2013}.
However, the applicability of ensemble forecasts are limited; for example, it is not currently possible to evaluate the effects of convergent predictions on total uncertainty of outcomes.
One of the greatest challenges inherent in both approaches above is estimating uncertainty, which is of utmost importance in a prediction context.
A more comprehensive understanding of uncertainty can guide biodiversity management and prioritize future data collection by identifying parameters that make large contributions to the total uncertainty in the model predictions \citep{McMahon2011}. 
In models that encompass multiple scales or levels of organization, it is particularly challenging to evaluate all sources of uncertainty, especially those related to model specification (\citealt{Calder2003} but see \citealt{Conlisk2013}).
Clearly, an approach that integrates predictions and uncertainty from multiple model types is needed \citep{Beck2012, Thuiller2013}. 
Such an approach would incorporate multiple modes of inference and integrate data from multiple sources and scales \citep{Levin1992, Peters2004, Thuiller2013}.

Here, we present a framework for modelling species range dynamics that aims to account for all the information available on a focal species, from a variety of sources and scales.
Contrary to hybrid and hierarchical models, the aim is not to link different sub-models into a single model, but to condition the predictions of a meta-model at the target scale (e.g., an entire species' range) with information from sub-models at a variety of spatial scales, allowing as much flexibility as possible regarding the type of information included. 
We use the power and generality of a hierarchical Bayesian framework, which allows us to include multiple data sources and modes of inference \citep{Clark2005, VanOijen2005, Clark2006, Hobbs2011, Hartig2012}. 
Another advantage of Bayesian methods is that uncertainty in model outputs is intuitive to interpret and reflects uncertainty at all levels of organization \citep{Clark2005, Cressie2009, Hobbs2011}. 
We illustrate our approach with two examples.
First, we use simulated data to present a complete example of the application of the framework to multiple information sources that are relevant at very different scales.
In a second example, we apply the framework to combine predictions of a correlative model relating the range of sugar maple (\emph{Acer saccharum}) to climate with the predictions of a phenological model, Phenofit, with the goal of improving uncertainty estimation when projecting model predictions to future climate.


%---------------------------------------------
%---------------------------------------------


\section*{Integrated modelling framework}
\aa{Integrated modelling framework and examples: I would just merge these two sections into: 'Examples of the integrated modelling framework'}

The key idea of our approach is to constrain the parameters of a correlative meta-model using predictions made by one or several mechanistic sub-models from other spatial scales.
\aa{Does the meta-model need to be correlative? Because later on you say that it 'could take the form of a correlative model' as implying that it could take other forms.}
The role of the metamodel is to integrate data at the same ecological scale as predictions. 
As an example, the metamodel could take the form of a simple correlative \ac{SDM}.
Ideally, each sub-model incorporates different data types and hypotheses that specify underlying ecological mechanisms contributing to the model output.
\tf{Check in examples if hypothesis are given}
\mt{they are there, though perhaps not so explicit}
The output of sub-models should provide additional predictions at a comparable ecological scale at which the metamodel operates, although it is not necessary for the sub-models to actually operate at the same scale.
For sub-models operating at a substantially different scale than the metamodel, scaling functions will be required; however, as we will show in the first example, these functions can be simple while still contributing useful information to the metamodel. 
All predictions may therefore be integrated in a single Bayesian framework when evaluating the parameters of the metamodel (see Box 1, Fig. \ref{fig:diagram}).
We present the complete framework in a hypothetical example where experimental results are  introduced as prior knowledge of the fundamental niche in a species distribution model (SDM).
A more formal presentation of the modelling framework appears in Box 1.
We follow with a second example where we apply our approach to integrate phenological information with occurrence data for a widespread North American tree.


\begin{figure}


%\documentclass{article}
%\usepackage{tikz}
%\usepackage{tikz-cd}
%\usepackage{amsmath}

%\usetikzlibrary{calc, shapes}
%\usetikzlibrary{shapes.geometric,shapes.arrows,decorations.pathmorphing}
%\usetikzlibrary{matrix,chains,scopes,positioning,arrows,fit}
%\begin{document}

%% --------------------------------------------------------------------------------------
%% ------part 1--------------------------------------------------------------------------
%% --------------------------------------------------------------------------------------

\begin{tikzpicture}
	
\matrix(m) [matrix of nodes, column sep=-0.5em,
	row sep=2em,
	minimum width=4em,
	minimum height=2em,
	column 1/.style={anchor=west, align=left, text width=10em},
	multi/.style={rectangle split,rectangle split parts=2}]
	{
	% zeroth line
	% blank space 
	&
	\textbf{Correlative Metamodel}
	&
	&
	\textbf{Mechanistic sub-model}
	\\
	% first line
	Data % m 1-1
	&
	$X_{M}, D_{M}$ % m 1-2
	&
	&
	$X_{S}, D_{S}$ % m 1-4
	\\
%second line
	|[multi]| Process % m 2-1 
	\nodepart{second}
	(prior and likelihood)
	&
	|[multi]|$p(\theta_{M})$ % m 2-2
	\nodepart{second}
	$p(X_{M} \mid \theta_{M}, D_{M})$
	&
	&
	|[multi]|$p(\theta_{S})$
	\nodepart{second}
	$p(X_{S} \mid \theta_{S}, D_{S})$
	\\

%third line
	
	Posterior
	&
	$p(\theta_{M} \mid X_{M}, D_{M})$
	&
	&
	$p(\theta_{S} \mid X_{S}, D_{S})$
	\\

%fourth line
	Prediction
	&
	$\psi_N = f(\theta_M, D_M)$
	&
	&
	$\psi_S = g(\theta_S, X_S, D_S)$
	\\

%fifth line
	Data
	&
	&
	&
	$\psi_S$
	\\
%sixth line
	|[multi]| Process
	\nodepart{second}
	(integrated likelihood)
	&
	$p(\theta_M \mid X_M, D_M)$
	&
	&
	$p(\psi_S \mid \theta_M)$
	\\


%seventh line
	Integrated posterior
	&
	&
	$p(\theta_M \mid \psi_S, X_M, D_M)$
	&
	\\

%eighth line
	Integrated prediction
	&
	&
	$\psi_I = f(\theta_M, D_M)$
	&
	\\
}; %end matrix

% The names of the nodes are automatically generated in the previous matrix. Since the
% matrix was named ``m'', all nodes have the name m-row-column

% metamodel
\draw [->] (m-2-2.south) -- (m-3-2.north);
\draw [->] (m-3-2.south) -- (m-4-2.north);
\draw [->] (m-4-2.south) -- (m-5-2.north);

%%mechanistic model
\draw [->] (m-2-4.south) -- (m-3-4.north);
\draw [->] (m-3-4.south) -- (m-4-4.north);
\draw [->] (m-4-4.south) -- (m-5-4.north);

%%separation line
\draw [line cap=rect, transform canvas={yshift=-1em}] (m-5-1.south west) -- (m-5-4.south east);
%(\linewidth-\pgflinewidth,0); 

\tikzstyle{opt}=[gray,dashed,rounded corners];
\draw [->,style=opt] (m-4-2.west) to[bend right=50] (m-7-2.west);
\draw [->,style=opt] (m-5-4.south) to (m-6-4.north);

%%integrated model
\draw [->] (m-7-2.south) -- (m-8-3.north west);
\draw [->] (m-6-4.south) -- (m-7-4.north);
\draw [->] (m-7-4.south) -- (m-8-3.north east);
\draw [->] (m-8-3.south) -- (m-9-3.north);


\end{tikzpicture}


\caption{The parameters of a correlative metamodel model (left column) are conditioned on the predictions of a mechanistic sub-model (right column).
The metamodel ($\theta_M$) operates at a single ecological scale and uses occurrence data (\(X_M\)) and explanatory variables ($D_M$) to produce a naive (i.e., not conditioned on sub-models) prediction $\psi_N$.
The mechanistic sub-model \(\theta_S\) includes data about the response (\(X_S\)) of lower-level behaviours of the system to explanatory variables ($D_S$). 
The models are integrated by calibrating $\theta_M$ to data ($X_M, D_M$) as well as the output of the sub-model ($\psi_S$). 
This is possible because predictions from the sub-model ($\psi_S$) emerge at the scale of the metamodel via a scaling function \(g(\theta_S, D_S)\).
The final prediction can be obtained by applying the integrated model to the original explanatory variables (as shown) or, if projection is desired, on some new set of explanatory variables (e.g., future climate).
This prediction incorporates multiple sources of information coming from several calibration datasets (i.e., $X_M, D_M, X_S, $ and $D_S$) as well as from multiple types of models (i.e., $\theta_M$ and $\theta_S$).
}
\label{fig:diagram}
\end{figure}


%---------------------------------------------
%---------------------------------------------

\section*{Examples}
\aa{Example 1: nothing to add, I think it is great as is. Figures 3 and 4 are very informative now. Great job! 
Example 2: it is nicely exposed, although the results don't seem too promising. You said you got new, better results, so that’s good news. ;-)}

Our modelling framework is best explained with concrete examples. 
In this section we first present how experimental results could be introduced as prior knowledge of the fundamental niche in a species distribution model (\ac{SDM}), and in the following we apply our approach to integrating phenological information with occurrence data for a widespread North American tree.
\ia{delete the above paragraph}
\mt{I tend to agree}

%---------------------------------------------
\subsection*{Example 1: Adding experimental evidence for the fundamental niche to a species distribution model} 

In this hypothetical example, we build a metamodel relating the distribution (i.e. occurrence probability) of an annual plant to coarse-scale climate with complementary information originating from a fine-scale experiment manipulating the precipitation regime.
The metamodel attempts to capture the realized distribution of a species, thereby encapsulating in a single correlative model the major physiological constraints and ecological processes constraining a species distribution. 
\mjf{How does it capture these constraints if it is a correlative model?}
However, for the purposes of forecasting, we would like to disentangle the fundamental response of a species to environmental variation from other processes in order to map the climatic envelope of where a species may be found in a natural setting. 
Prior information of the physiological constraints affecting species distribution is sometimes available, but usually too incomplete to perform a reasonable comparison with the realized distribution.
For instance, as in this example, a species distribution might be constrained by both temperature and precipitation, but experiments were conducted only over a precipitation gradient at one temperature. 
\dm{May want to provide a bit more justification as to why this particular hypothetical example is useful? When I think of experimental data I think you are meaning lab-controlled data which is quite rare I think?}
We apply our framework here to such heterogeneous sources of information to reduce the bias in parameter estimation and more adequately represent uncertainty. 
Here, we focus only on the specification of the modelling framework; complete procedural details and code for generating the data sets and executing the model are provided as supplemental information.

We consider data collected from a species' historical distribution, where the goal is to predict the distribution following a substantial reduction in precipitation. 
We will consider \(X_M\) (see Box 1, Fig. \ref{fig:diagram}) to be a vector of $n$ observations that takes the value of one to indicate presence and zero for absence:
\begin{equation}
X_M = \{x_{M,1}, x_{M,2}, \ldots, x_{M,n}\}
\end{equation}
\dm{Just noting my initial comment that even absence data can be of course tricky eg structural zeros; a cryptic species; looked at the wrong time of day; some disturbance happened that year….etc….etc.  I note your comment in the next sentence which is good}
We further assume the data are relatively high-quality, providing coverage over a wide region and covering various climatic conditions (Fig. \ref{fig:ex1_sampling}a). 
The model is evaluated by relating these observations to environmental data \((D_M)\), which for the sake of the example consists of mean annual temperature $T_M$ and annual precipitation $P_M$. 
For simplicity, we use a simple logistic model for the metamodel \((\theta_M)\) with a second order effect of temperature and precipitation. 
This naive model (i.e., the metamodel with no constraints from sub-models) predicts  occurrence probability (\(\psi_N\)) as a function of the environment:
%-----
\begin{equation}
\begin{aligned}
	\psi_N &= f\left(\theta_M, D_M \right) \\
	&= p \left (X_M = 1 \mid \theta_M, T_M, P_M \right) \\
	&=\text{logit}^{-1}\left( \mathbf{\Theta_M} \mathbf{D_M} \right)
\end{aligned}
\end{equation}
%-----
where \(\mathbf{\Theta_M}\) is the parameter vector of the model \(\theta_M\), \(\mathbf{D_M} \) is the covariate matrix (i.e., \(T_M, P_M\), with the first column taken to be unity to allow an intercept to be fit), and \(\text{logit}^{-1}\) is the inverse of the logit function.
We can fit this model in a Bayesian framework to allow for easy integration of models (as we will show later).
In this context, the goal of modelling is to estimate the probability distribution of \(\theta_M\), the model parameters, given the observed data \((X_M, T_M, P_M)\), which is given by the proportional form of Bayes's Theorem \citep[for readers unfamiliar with general concepts in Bayesian inference, we recommend][for a concise introduction]{Link2010}:
\fd{Bonne idée d'avoir introduit cette proposition ici.}
\dm{add citation: McCarthy M.A. 2007 Bayesian methods for Ecology. Cambridge University Press 296p}
\dm{Mick McCarthy might be a good suggestion for a reviewer (I know him – an Aussie)}
%-----
\begin{equation}
\label{eq:ex1_bayes}
	p\left (\theta_M \mid X_M,T_M,P_M \right ) \propto 
	p \left(X_M \mid \theta_M, T_M, P_M \right)
	p \left(\theta_M \right)
\end{equation}
%-----
where \(p\left(X_M \mid \theta_M, T_M, P_M \right)\) is often referred to as the \emph{likelihood} of the data (\(X_M\)) given the model (\(\theta_M\)), \(p\left(\theta_M \right)\) is often referred to as the \emph{prior distribution} of \(\theta_M\), and the goal of modelling is to estimate \(p\left (\theta_M \mid X_M,T_M,P_M \right )\), the \emph{posterior distribution} of \(\theta_M\), which gives the probability that \(\theta_M\) takes particular values, given the observed data.

%==================
% FIGURE

\begin{figure}[tb]
	\includegraphics{ex1_sampling.pdf}
	\caption{Two simulated data sets used to illustrate the model integration framework.
	(a) Presences (circles) or absences (x's) of the species in ecological space, where precipitation ranges from 0--1.
	(b) Growth rate ($r$) as a function of manipulations to the precipitation regime (whiskers show $\pm$ 1 SE), with a larger range for precipitation (i.e., \(-\)1--1).
	The dashed line shows the threshold above which the species net growth rate is positive (implying presence).
	Axis scales for temperature and precipitation are arbitrary, but note the different scales on the horizontal axes.}
	\label{fig:ex1_sampling}
\end{figure}

% ERUGIF
%==================

Thus far, we have considered only a single source of information to fit this model, and therefore the prior distribution \(p\left(\theta_M \right)\) from Eq. \ref{eq:ex1_bayes} is uninformative.
As a secondary source of information, we will consider an experiment relating the population growth rate of the plant to manipulations to the precipitation regime, with results (but no raw data) available from the literature (Fig. \ref{fig:ex1_sampling}b). 
Furthermore, no information is available regarding the temperature regime for the experiment.
Transplant experiments that evaluate performance beyond the range of a species are common, and represent a plausible scenario for model integration \citep{Hargreaves2014}.
According to niche theory \citep{Holt2009}, the fundamental niche corresponds to the set of environmental conditions where the per capita intrinsic growth rate $r$ is positive.
This concept gives us a reasonable model to fit the scaling function $g$ for our sub-model (Fig. \ref{fig:diagram}).
If we hypothesize that the errors from Figure \ref{fig:ex1_sampling}b \( \left(\sigma_{S} \right) \) are normally distributed, then for observation $i$, we can interpret the probability of presence \( \left(\psi_{S,i}\right)\) as the probability that the observed growth rate \(X_{S,i}\) is positive:
\begin{equation}
	\psi_{S,i} = \int_0^\infty N \left(X_{S,i}, \sigma_{S,i} \right)
\end{equation}
where \(N\) is the Normal density function.
We can then estimate the posterior distribution for the sub-model by fitting the relationship between \(\psi_S\) and precipitation \( \left( P_S \right) \) using Bayesian beta regression \citep{Ferrari2004}:
%-----
\begin{equation}
\label{eq:ex1_thetas}
	p\left (\theta_S \mid \psi_S,P_S \right ) \propto 
	p \left(\psi_S \mid \theta_S, P_S \right)
	p \left(\theta_S \right) \\
\end{equation}
%-----

Although the two data sets were collected at considerably different scales, we now have sub-model predictions arising from a fine-scale experiment that are relevant at the scale of the metamodel (i.e., the probability of presence at a given precipitation). 
This scaling is quite simplistic, and would never be used to predict a species' range using this mechanistic sub-model in isolation.
Despite the strong theoretical foundation (i.e., niche theory) linking the observations to presence-absence at large scales, the upscaled sub-model (i.e., the transformation of \(X_S\) and \(\sigma_S\) to \(\psi_S\)) is built on the strong hypothesis that the species will be present when $r>0$, neglecting other population dynamics constraints such as Allee effects and metapopulation dynamics. 
The fundamental niche is also incomplete, as we lack environmental variables such as temperature, as well as all other ecological processes responsible for the realized distribution.
As such, it would be unwise to expect predictions from this model alone to resemble the actual distribution of the species; as a mechanistic model, it is simply too incomplete to predict distribution.
As we will show, however, the information from this sub-model, when applied as a constraint on the metamodel, can result in improved predictions that incorporate the information within each model.

We accomplish model integration by treating the posterior of \(\theta_S\) as prior information about some of the parameters of \(\theta_M\) (i.e., parameters related to precipitation), expanding Equation \ref{eq:ex1_bayes} to incorporate the new information from the sub-model:
%-----
\begin{equation}
	\label{eq:ex1_integrated}
	\overbrace{p(\theta_M \mid X_M, T_M, P_M, \theta_S, \psi_S, P_S)}^\text{integrated posterior}
	\propto
	\overbrace{p\left (\psi_S \mid \theta_S,P_S \right )}^{\substack{\text{new information} \\ \text{from sub-model}}}
	\overbrace{p \left(X_M \mid \theta_M, T_M, P_M \right) P \left(\theta_M \right)}^{\substack{\text{naive metamodel} \\ \text{posterior}}}
	\overbrace{p \left(\theta_S \right)}^{\substack{\text{prior for} \\ \text{sub-model}}}	
\end{equation}
%-----
As before, the metamodel \(\theta_M\) can be used to predict the species distribution \((\psi_I)\).
However, these predictions will not reflect only the presence-absence samples in \(X_M\), but will also reflect the information from \(\theta_S\), including all of the data sources used to produce this sub-model.
In other words, with this evaluation, the set of parameters $\theta_M$ is most likely when it agrees with both the original data and predictions from the sub-model. 
Note that in this particular example, because the sub-model is only contingent on precipitation, there is no integration on temperature. 
Finally, we note the presence of marginal distributions for both models (i.e., \(p(\theta_M)\) and \(p(\theta_S)\)).
These can be informative (e.g., incorporating further prior information or the predictions of additional sub-models), semi-informative (e.g., to provide greater weight to more informative models), or uninformative.
\fd{A word should be said about how the likelihood function balance the two sources of information for (Psi)M.  (How greater weight can be given to a more informative sub-model).}
For purposes of this example, we used these terms to reduce the precision of the second model to reflect the uncertainty of the modelling process (due to the lack of additional mechanisms in the second model, as mentioned above).
\fd{MAYBE NOT EXACTLY THERE:  "Moreover, convergent or divergent sub-models' response in ovelapping scale domains can  help in detecting crucial ecological processes."}

We performed this procedure with our hypothetical data sets.
When comparing the three models (naive metamodel, mechanistic sub-model, and integrated metamodel), we observed extreme uncertainty in the first model when projecting beyond the range of the original data.
Unsurprisingly given the quality of the data, the sub-model was highly precise, providing a fairly strong constraint when producing the integrated model.
The result was an integrated prediction that reflected the shapes of both models and showed considerably reduced uncertainty (Fig. \ref{fig:ex1_precip}).
At the scale of the metamodel, considering both temperature and precipitation, we observed similar results, with reduced uncertainty in the predictions over the domain not covered by the presence-absence data (Fig. \ref{fig:ex1_map}).


%==================
% FIGURE

\begin{figure}[tb]
	\includegraphics{ex1_precip.pdf}
	\caption{Results of integration showing the probability of presence (\(\psi\)) when considering only precipitation.
	(a) Naive model, using only presence-absence data. Uncertainty increases dramatically when attempting to project beyond the scope of the source data (represented by the horizontal line above the bottom axis).
	(b) Mechanistic model, using observations of an experiment to infer probability of presence. Predictions are highly precise due to high quality source data, but are likely to be overly optimistic given the simplicity in assuming only precipitation determines probability of presence (see text).
	(c) Integrated model, showing predictions that are intermediate between the two sub-models and uncertainty that is reduced compared to (a).
	Uncertainty is represented as dashed lines, showing the limits of 90\% Bayesian credible intervals.}
	\label{fig:ex1_precip}
\end{figure}
% ERUGIF
%==================


%==================
% FIGURE

\begin{figure}[tb]
	\includegraphics[width=5.25in]{ex1_map.pdf}
	\caption{Maps showing the predicted probability of presence (\(\psi\); (a) and (c)) and the standard deviation of \(\psi\) ((b) and (d)) for the naive and integrated models.
	Historical (e.g., where presence-absence samples were available) and predicted future precipitation regimes are shown below the horizontal axes.
	Uncertainty in the naive model was extreme when predicting beyond the historical precipitation regime.
	Uncertainty in the integrated model was considerably reduced, reflecting the additional information provided by the mechanistic model.}
	\label{fig:ex1_map}
\end{figure}
%
% ERUGIF
%==================
\dm{FIG 4: Again just a comment …one could discuss what does this probability of presence really mean?  Presence with a self sustaining population. Someone could plant a specimen outside its historic natural range and expect it to live.  I woder if rather than naïve the term "simple" would be better? I have substituted for that word in some places …Aren’t we all naïve ??!!}

Once the technical challenges of evaluating the parameters are overcome, the integration has several advantages over the both the naive parameterization of the metamodel and the mechanistic sub-model. 
In the situation where both models strongly agree on the relationship between water availability and occurrence probability, then we should expect significantly reduced uncertainty in that domain (Figs. \ref{fig:ex1_precip}, \ref{fig:ex1_map}).
Because the metamodel includes both temperature and precipitation, any reduction in the uncertainty with respect to precipitation might also influence the evaluation of the relationship with temperature (reducing bias in the case where there is a correlation between T and P in the data). 
However, uncertainty could increase in the case of model disagreement.
For instance, the relationship between precipitation and presence in the naive metamodel could be caused by a spurious relationship (e.g. spatial autocorrelation caused by historical contingencies), leading to a false confidence in the species distribution over precipitation gradients. 
The information from the sub-model will thus bring back the right amount of uncertainty in the relationship.

\subsection*{Example 2: Constraining an SDM using phenological information}
For the second example, we consider the problem of predicting the future distribution of a species following climate change.
There is considerable interest in comparing correlative and mechanistic projections with respect to climate change \citep{Morin2009}, and correctly characterizing uncertainty is a critical aspect of this problem \citep{Cheaib2012}.
Despite being a relatively common application of \ac{SDM}s \citep{Guisan2005}, projecting models parameterized with modern climate data to future climate scenarios is problematic for a number of reasons previously discussed \citep{Araujo2006, Austin2011}.
We used our metamodeling framework to inform a climate-based SDM with information obtained from Phenofit, a mechanistic model that predicts a species' probability of presence as a function of the suitability of the environment given the species' phenology \citep{Chuine2001, Morin2009}.
We use data from \citet{Morin2009} for sugar maple (\emph{Acer saccharum}), an economically and ecologically important species occurring in eastern North America.
Here we describe briefly the dataset, methods, and the results of the analysis.
Complete details, including all data and scripts to reproduce the analysis, are included as supplemental information.

\defcitealias{IPCC2001}{IPCC, 2001} %% this is needed to make the organizational citation in the following paragraph work correctly
For the metamodel, we used a dataset of 1013 sugar maple presence records at 0.5 degree resolution, with 498 corresponding pseudo-absence records sampled from North America.
\mt{update numbers with final versions, and indicate that they are true absences}
We used a binomial GLM to relate the suitability for sugar maple to 5 climate variables, scaled to the same resolution (see SI for a description of the climate variables used).
fd{Present the abbreviation for SI= Supplemental Information earlier in the text (at the end of the previous paragraph, for example.)}
\wt{We could also have used something more complex such as a GAM or boosted regression trees. }
\mt{Perhaps note this; we selected this model for simplicity of explanation and implementation, but more flexible functions are possible}
For projection, we used predictions for 2080 from the HadCM3 GCM \citep{Pope2000} driven by the A2 emission scenario (REF--Nakicenovic), and used the parameters of the GLM to forecast suitability into the future (i.e., \(\psi_N\)).
\wt{citation:
Nakicenovic, N. \& Swart, R. (ed.\^eds) (2000) Emissions Scenarios: A Special Report of Working Group III of the Intergovernmental Panel on Climate Change. Cambridge University Press, Cambridge.}
\tf{Is it an optimistic or pessimistic or medium scenario? 
Would it be interesting to compare two extreme scenarios? Would it change something to the demonstration of this manuscript?}
Phenofit produces predictions at continental scales; thus the scaling function \(g\) was incorporated within Phenofit itself and we performed model integration directly on the outputs.
To perform the integration, we used \ac{MCMC} to condition the parameters of the metamodel simultaneously on two likelihood functions.
The first is the naive SDM, where the likelihood of a ``success'' in the presence/pseudo-absence dataset was a Bernoulli density with the (logit-transformed) probability expressed as a linear function of the climate predictors.
The sub-model likelihood was the Beta density of the observed Phenofit prediction for the present climate given the predicted suitability under the naive model. 
Thus, the posterior probability of the metamodel was:
\begin{equation}
\label{eq:integrated2}
	p( \theta_M \mid X_M, D_M, X_S, \theta_S )
	\propto 
	p( X_S \mid \theta_M, \theta_S )
	p( Y_M \mid \theta_M, D_M ) 
	p( \theta_M )
	p( \theta_S )
\end{equation}
where \(\theta_M\) is the metamodel, 
\(X_M\) is the vector of presences and pseudo-absences, 
\(D_M\) is the matrix of climate predictors,
\(X_S\) is the prediction from Phenofit (in this case a single deterministic value, because we did not incorporate errors from the Phenofit modelling),
and \(\theta_S\) is the model relating the Phenofit predictions to the metamodel.
\fd{Shouldn't be Psi\_S here as we speak of prediction of the sub-model?}
Note that here, the integration is performed on the \emph{predictions} of the sub-model directly, rather than on compatible parameters as in Example 1. Thus, the metamodel must appear in the sub-model likelihood.

The naive model, despite being relatively simple (xx climate variables with xx total parameters), was highly fit to the data and consequently showed very high suitability scores near the core of the contemporary range with very little uncertainty (Fig \ref{fig:ex2}C, E).
Similar patterns were repeated for analogous climates under the future scenario (Fig \ref{fig:ex2}D, F).
However, standard errors were extreme for the range margins where suitability scores were intermediate.
In comparison, Phenofit suitability scores were more moderate, and Phenofit predicted less northward movement of sugar maple under the future scenario \citep[Fig. \ref{fig:ex2}A, B;][]{Morin2009}.
The integrated model demonstrated generally lower suitability scores throughout both the present and projected ranges, and thus represents lower confidence in the presence of sugar maple at any given location in its future projected range (Fig \ref{fig:ex2}G, H).
Standard errors at range margins were significantly lower (as these are regions where both the naive model and phenofit were in agreement about intermediate suitability scores), whereas standard errors were slightly higher at the core range (where Phenofit predicted lower suitability than the naive model) (Fig \ref{fig:ex2}I, J).

%==================
% FIGURE

\begin{figure}[tb]
	\includegraphics[width=6in]{ex2.pdf}
	\caption{Results of model integration for sugar maple, \emph{Acer saccharum}.
	The mechanistic sub-model Phenofit predicts present (A) and future (B) suitability using phenological information, but does not provide estimates of uncertainty.
	Predictions from a naive metal-model for the present (C) were quite similar to the sub-model, but future predictions (D) were somewhat different, and uncertainty was extreme at the predicted range margins for both present (E) and future (F).
	Applying the integration results in reduced suitability overall, possibly due to overfitting regions where the models disagree, but patterns in relative suitability largely reflected information from both models for both present (G) and future (H) climate.
	Integrated model uncertainty for both present (I) and future (J) climates is reduced near range margins but slightly increased in areas where uncertainty was low in the naive model.
	}
	\label{fig:ex2}
\end{figure}
%
% ERUGIF
%==================
\tf{Fig 5: Is it realistic to make predictions as South as Mexico or In Cuba? 
A question arises for me: to which spatial enveloppe should these models be used? Worldwide? Smaller enveloppe?
This opens questions about species interactions. The predictions made are for sugar maple only without taking into account other speices. Sugr maple is present in EastAmrica. Is there another species which is only present in WesternAmerica ? Do we known how both species interact and who will “win” the competition game?
I guess that the approach proposed is dedicated to answer these questions and that the figure are only here to present the distribution of sugar maple when only climate is changing.
}

For modelling the contemporary range of sugar maple, both Phenofit and the naive model perform very well relative to the integrated model, espeicially when capturing the range margin (Fig \ref{fig:ex2}A, C, G).
Conventional SDMs are quite good at producing empirical models of species distributions under present climatic regimes, and the naive model here could be further improved with the addition of more parameters or the inclusion of ensemble forecasting (See SI for model-fitting details).
However, Phenofit and the naive model differ substantially in their predictions of the future distribution of the species.
The integrated model presents a compromise between the two predictions. 
The core range of the species under the integrated model generally reflects the regions of highest overlap between Phenofit and the naive model.
Regions surrounding the core range generally show low suitabilities, but higher than those under either sub-model, demonstrating reduced certainty about where the species will be absent in the future.




%---------------------------------------------
%---------------------------------------------

\section*{Discussion}


\subsection*{Comparison with other methods}
The methods provided here expand upon the motivation of hybrid models to develop a more robust and unifying approach for ecological models in a predictive context while striving to overcome some of the limitations characteristic of other integrated approaches. 
Attempting to incorporate underlying ecological processes so that SDMs are not oversimplified by combining different model parameters or scales can be challenging using these hybrid approaches. 
In particular, it can often be difficult to identify parameters that can be used to connect different modeling frameworks and what parameters can produce meaningful responses \citep{Thuiller2013}. 
The methods presented in this study aim to overcome some of these limitations through the use of Bayesian statistics. 
Bayesian methods produce results as an entire distribution instead of a single point, allowing for a comprehensive understanding of the uncertainty of estimates \citep{Link2010}. 
The estimation of model parameters (and predictions) as distributions can also help decrease overparameterization that can be a concern for hybrid models. 
Bayesian frameworks also provide a natural framework for the incorporation of prior information. 
The inability or difficulty to include information from experimental studies or ecological processes at lower scales can be a possible drawback of hybrid models \citep{Thuiller2008, Smolik2010}. 
Even when a link can be formed to include this information, it often is oversimplified to simplify computation \citep{Gallien2010}. 
Finally, Bayesian methods inherently allow for feedbacks or interactions between sub models, which may portray a more realistic response to many environmental dynamics where factors may simultaneously influence the change of one another.

Our approach is an original application of Bayesian multilevel modeling, and can be considered a logical extension of other Bayesian approaches developed to deal with processes that occur at multiple scales while using several models simultaneously. 
In particular, this development has certain similarities with Bayesian model averaging, Bayesian calibration of process-based models and hierarchical Bayesian modeling. 
Bayesian model averaging aims to combine several alternative models to obtain better predictions while taking into account parameter uncertainties \citep{Hoeting1999}, and has been applied numerous times in ecology where the mechanisms underlying a complex phenomenon are often unknown \citep[e.g.][]{Wintle2003, Link2006}. 
Bayesian model averaging considers models operating at the same scale and the posterior distribution is usually determined with Gibbs sampling. 
Bayesian calibration of process-based model focuses on uncertainty of the parameter values in the process-based models, in this case the values of the parameters are calibrated by the model output \citep{VanOijen2005, VanOijen2011, Hartig2012}. 
In contrast with Bayesian model averaging and calibration techniques, our approach handles data and models operating at different hierarchical scales, and uses process-based models to constrain the shape of a generally more correlative metamodel.
Similar hierarchical approaches have been applied to diverse fields, including engineering \citep{Booth2013}, hydraulic conductivity models \citep{Dostert2009, Efendiev2005, Efendiev2005a}, plant physiology \citep{Ogle2008, Ogle2009}, and climate and atmospheric modeling \citep{Zimmerman2005, Mcmillan2010, Kang2012}.
Although our method as presented is distinct from Bayesian model averaging, it is fully compatible with these methods.
Multiple metamodels could be averaged in this way if a consensus prediction from multiple metamodels (containing, e.g., different sub-model sets or different metamodel formulations).
Furthermore, the metamodel can easily be incorporated into a broader multimodel inference scheme.
For example, the metamodel can be used as one input in an ensemble forecasting scheme that also incorporates other common SDM formulations (e.g., Random Forest, MaxEnt).

\fd{FD will work on this part: \\
It would be good to compare this approach with the kind of one develop by Moorcroft et al. 2001 : Moorcroft, P. R., Hurtt, G. C., \& Pacala, S. W. (2001). A method for scaling vegetation dynamics: the ecosystem demography model (ED). Ecological monographs, 71(4), 557-586.\\
%Medvigy, D., Wofsy, S. C., Munger, J. W., Hollinger, D. Y., \& Moorcroft, P. R. (2009). Mechanistic scaling of ecosystem function and dynamics in space and time: Ecosystem Demography model version 2. Journal of Geophysical Research: Biogeosciences (2005–2012), 114(G1).\\
Fisher, R., McDowell, N., Purves, D., Moorcroft, P., Sitch, S., Cox, P., \& Ian Woodward, F. (2010). Assessing uncertainties in a second-generation dynamic vegetation model caused by ecological scale limitations. New Phytologist, 187(3), 666-681. 
\\
These models are multilevel and use scaling functions using statisitical moments series.
}

\subsection*{Advantages of Model Integration}
\aa{Discussion: the section ‘Advantages of model integration’, which I wrote, and ‘Applicability of the method’ have several points in common. Maybe they could be merged into a single section: ‘Applicability and advantages of model integration’. Maybe, it should be discussed more extensively how to face the situation in which predictions of correlative model and sub-models are so different that the integrated model performs poorer than the two (I mean, does that mean that the approach is not valid in those cases?)}
\fd{I see that paragraph coming closer to the end of the paper in the accesibility section, maybe}
Models are important tools that are increasingly used by land managers to face challenges associated to decision-making \citep{Guisan2013}
A growing number of models are available, but they can provide diverging answers to very similar questions due to the differences in their assumptions and methodology. 
This can create confusion and even some mistrust towards models, and as a consequence some managers may be discouraged to incorporate model results in their management plans. 
By integrating different types of models---and their outputs---into a common framework, we believe that our approach can contribute to overcome the gap between modelers and practitioners and thus promote wider use of models to support decision-making.
\ia{This is repetitive with the applicability section. I suggest that you stick here on the modeling advantage of your approach and keep the applied advantage for the other section}

Model uncertainty is another key factor affecting applicability of model outputs \citep{Addison2013}.
One of the main strengths of our approach is that it allows for a clear, transparent identification of uncertainties and how they are transmitted throughout the modeling process. 
Transparency in uncertainty can be considered as a sort of sensitivity analysis, in which areas with large uncertainty can be detected and new experimental research or additional data collection can be designed (e.g., Example 1, Figs. \ref{fig:ex1_sampling}, \ref{fig:ex1_precip}).
The new knowledge resulting from this research can be readily incorporated into the metamodel, allowing for an iterative learning process that will undoubtedly contribute to reduce uncertainty in predictions for a wide range of environmental conditions. 
Moreover, our framework embraces change as a fundamental process and is able to adapt and respond to it. 
The ease of incorporating new knowledge to the modeling framework (including new theory or the result of management and experimental efforts), will allow for a rapid adjustment of the predictions and the incorporation of the most recent available knowledge into management plans \citep{Keith2010}.
In an era of continuous and unprecedented change, adaptive approaches such as the one presented here are often highlighted as a pressing need in order to develop strategies to promote ecosystems that are both feasible and resilient \citep[Fig. \ref{fig:management};][]{Seastedt2008, Mori2013}.
\wt{
Perhaps we could also be more precise here. Give examples. Since you are working with forest data, an easy step forward would also be to calibrate growth curve of most tree species in function of both temp and precipitation, even a competition model (see Kunstler et al. 2011 J. of Ecology), that can then be integrated into SDMs. 
Similarly you may also discuss about the NEON and LTER long term sites that could certainly give similar information than the one used in Ex 1. I think we should strenghen the different possibilities to make sure users find the approach useful. It remains to vague to me at this point. 
If you think about the Cheaib et al. paper, you may also suggest that your approach could be run over all different model outputs to integrate thema all together. 
In a way that should re-inforce the link between modelers and long term experimental ecologists that did not really discuss so far since it was difficult to integrate the two informations. 
}


\subsection*{Challenges} 
Although our approach is highly flexible and can be applied in a number of situations, there are some challenges to successfully using the framework.
As with any modeling effort, good model specification with strong links to theory are essential \citep{Austin2007}.
Poorly specified models will produce outputs that are uninformative or misleading, and model integration is not a cure for these problems.
In the worst case, constraining a metamodel with a poor sub-model can result in outcomes that are worse in terms of bias and uncertainty than those produced by a naive model.
We expect model selection will play an important role in applications of our framework, and such schemes can be incorporated relatively easily \citep{Madigan1995, Wasserman2000, Tenan2014}.

In a similar fashion, the quality and availability of data impose a significant constraint on the number and type of models that can be implemented.
The capacity to implement a model is very low if data requirements are prohibitive. 
Adequate coverage of explanatory variables is a significant obstacle, and exploratory analyses can be a significant aid in understanding how data coverage impacts resulting predictions \citep{Mckenney2002}.
Integration can ameliorate these issues to some extent by using supplemental information (and conceptual advances) in additional sub-models where coverage is weak (e.g., Example 1, Figs. \ref{fig:ex1_precip}, \ref{fig:ex1_map}).
For example, \citet{Freckleton2009} estimated that data are too scant to successfully develop highly mechanistic models predicting weed population numbers. 
A strong asset of our approach is that it can be used without the full suite of data that would be required to run a fully mechanistic model. 
Given that the metamodel is correlative, it can be effectively implemented with, e.g., only presence-absence data, or, in the case where true absences are difficult to obtain, with presences and pseudo absences (provided sufficient care is used in interpreting the results of such a model).
Then any additional mechanistic data that are available will enhance predictions by constraining outputs of the metamodel. 

Finally, determining the functions to use to express the likelihood of the sub-models given the metamodel (i.e., Eqs. \ref{eq:ex1_integrated}, \ref{eq:integrated2}) is a critical point.
In the context of integrated models, the challenge is three fold: (i) which spatial and temporal scales (i.e. which processes), are to be considered, (ii) how to build a scaling function that is consistent with the metamodel and (iii) how error and uncertainty are propagated from the scales of the sub-models scale to that of the metamodel. 
Although we argue that our proposed framework is able to easily deal with very different scales and that the Bayesian framework allows for an efficient integration of uncertainty throughout all scales considered, the building of scaling function deserves further investigation. 
We have shown in our examples that even simple scaling functions can provide reasonable constraints on the metamodel. 
However, with the objective to take into account all know processes and models, it is likely that the modeling process may include multiple scaling functions, each at different scales. 
Indeed, if species distributions are a function of, e.g. poopulation growth rate \citep{Sykes1996, Guisan2000}, they will involve processes at the individual (e.g. competition) or cellular (e.g. photosynthesis) scales. 
Such very large differences in spatial scales imply that simple functions will be inadequate, requiring more traditional upscaling methods. 
Therefore, intermediate models and processes will need to be introduced, and this will require additional parameters, assumptions, and data, and will add uncertainty to the final integrated model.



\subsection*{Applicability of the method}
\defcitealias{TERN2013}{TERN, 2013} %% this is needed to make the organizational citation in the following paragraph work correctly
Integrated approaches have gained momentum in recent years, with integrative science being featured as a central theme for several science-based governmental organizations around the world \citep[e.g.][\citetalias{TERN2013}]{Bernier2013}. 
Incorporating information from multiple sources, particularly with respect to uncertainty,  fosters a connection between scientifically-generated knowledge and policy, and is therefore an important tool for adaptive management \citep[][Fig. \ref{fig:management}; ]{Rehme2011}.
Such approaches are needed in formulating management plans for vulnerable species and ecosystems to avoid basing decisions on too-narrow subsets of the available information \citep{Dawson2011}.
The flexibility of our approach may also represent an advantage for management by easing the integration of specific decision making criteria (e.g. desired grain and extent of outputs) into model development. 
Successful use of an integrated modeling approach will always remain dependent on an intimate understanding of the decision-making process by modelers, emphasizing the importance of close collaboration between with practitioners at all stages of model development \citep{Guisan2013}.

The transparency of our approach is also a clear advantage. 
To be useful, models should be transparent analytical tools, not black boxes \citep{Addison2013}. 
A key criterion for model applicability is the explicit and detailed communication of specific model objectives, characteristics, limits, and uncertainties, as well as its ecological foundation \citep{Guisan2013}.
Integrating sub-models allows for clear specification of desired model outputs (via the specification of the metamodel) while easily retaining important ecological objectives (via careful specification of sub-models).
Insufficiently documented models are difficult to assess in real world situation, and therefore only have limited relevance for decision support \citep{Addison2013, Guisan2013}.
Model workflow documentation becomes more crucial in the case of integrated modeling approaches, which incorporates information from various scales and resolutions within a number of aggregated sub-models. 
A transparent and well documented workflow describing the process of model integration ensures reproducibility and applicability (Fig. \ref{fig:management}). 

\subsection*{Accessibility}
\mjf{Accessibility of what? Modeling? Data?}
\mt{maybe remove the label and relocate to challenges}
Results of models predicting species ranges are increasingly used by resource managers, conservation biologists, or forest ecologists for formulating or adjusting recommendations. 
The utility of such models depends on their ability to help evaluate events beyond the bounds of the available data, including in future situations constrained by climate change. 
In this context, practitioners may judge model usefulness on two criteria: (1) Does the model provide reliable predictions at the needed time and space scales, and (2) can the model be implemented given the available data and technical expertise? 
Since most decision makers work at local space scales and follow specific time frames, modelers face the obvious challenge of providing detailed outputs while preserving implementation capacity.
In terms of model outputs, our framework is transparent in terms of uncertainty by virtue of providing easily interpretable probability distributions for model parameters and predictions.
However, developing the models requires careful model specification, understanding of applied Bayesian methods, and, in some cases, extensive programming.
In many cases, off-the-shelf software \citep[e.g.,][]{R, RJAGS} can adequately express the model likelihoods with minimal programming, but more complicated models will require the development of custom software.
Thus, this framework will require significant investment in developing customized code for the samplers in order to actually estimate parameters, which may be technically challenging for some practitioners and may limit the adoption of these methods by both practitioners and decision-makers.
However, the same was true of all statistical techniques or modeling approaches when they were initially developed. 
In addition, some of the computation costs that were associated with many techniques have now vanished, and even the conceptual challenges associated Bayesian statistics are being reduced as they gain recognition in the scientific literature. 
We therefore argue that our proposed approach as a strong potential for direct use in our real world where climate is quickly changing and conservation practices must be adjusted accordingly.
We further argue that forging stronger collaborations between modelers, decision-makers, and practitioners will improve the incorporation of this and other new methods into applied ecology and conservation.

\subsection*{Future directions}
Hierarchical Bayesian modeling requires data at every scale to fit the parameters required to constrain one level with another.
Moreover, beyond the needs for additional data, more knowledge of the important processes driving species range limits is needed to allow for extrapolation beyond the scope of the original datasets to address future changes due to climate and land-use changes.
This knowledge can be obtained through a series of experimental manipulations and by substituting space for time data along environmental gradients.
When experiments are possible, gaps in knowledge can be inferred using surrogate variables, theoretical models and spatially explicit models to explore the range of outcomes possible \citep{Fortin2012}. 
Another way to gain knowledge is to simplify species complexity into a series of states and to use state-transition models (e.g., Markov chain, semi-Markov chain, matrix projection model, integral projection model) which require more theoretical or coarser knowledge of species responses to biotic and abiotic changes.
Although fewer states allow us to better capture the essence of species responses to environmental conditions, knowledge is still required to model feedback effects across processes and spatio-temporal scales.
Hence we need to determine which processes are the most important at a given scale and how to weight their effects while scaling-up.
Our meta-modeling approach can act as a springboard for incorporating imperfect knowledge of how multi-scale processes influence species ranges.
By incorporating this knowledge into species distribution models, we can hopefully both reduce bias and more accurately assess model uncertainty when forecasting future changes to species ranges.

%---------------------------------------------
%---------------------------------------------

\input{model_description}

%---------------------------------------------
%---------------------------------------------


%% ----------------------------------
%
%     REFERENCES
%
%% ----------------------------------

\renewcommand\refname{Literature Cited}
\bibliography{model_integration}{}
\bibliographystyle{model_integration}



%% ----------------------------------
%
%     FIGURES
%
%% ----------------------------------




\end{document}